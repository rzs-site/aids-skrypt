\chapter{Algorytmy sortowania}

\epigraph{Jeśli coś~jest głupie i~działa, to~nie jest głupie.}{autor nieznany}

Uważa się, że~jednym z~najbardziej fundamentalnych problemów 
leżących u~podstaw informatyki jest problem sortowania. W~tym rozdziale 
zostaną przedstawione jedne z~najbardziej popularnych algorytmów sortowania.
Zanim jednak podejmiemy~się prób opisu tych algorytmów, przyda~się teoretyczne wprowadzenie.

\section{Wprowadzenie}\label{sec:sorting:intro}

Wprowadźmy ważne definicje.

\begin{definition}[Tablica]\label{def:array}
    Niech \( D \) będzie niepustym zbiorem. Tablicą
    \( n \)-elementową (\( n \in \NN \)) o elementach
    ze zbioru \( D \) nazywamy skończony ciąg
    \( A = (a_1, a_2, a_3, \dotsc, a_{n - 1}, a_n) \in D^n \).
\end{definition}
\begin{remark}
    W~przypadku podciągu \( (a_i, a_{i + 1}, a_{i + 2}, \dotsc, a_{j - 1}, a_j) \) 
    stosujemy oznaczenie \( A \left[ a \isep b \right] \),
    przy~czym \( 1 \le i \le n \), \( 1 \le j \le n \) 
    oraz~\( i \le j \).
    W~szczególności \( A \left[ 1 \isep n \right] \) 
    oznacza całą tablicę \( n \)-elementową.
    Stosować będziemy również oznaczenie na~\( i \)-ty 
    element tej~tablicy jako \(A \left[ i \right] \).
\end{remark}
Mając zdefiniowaną tablicę możemy sformułować następujący problem.
\begin{problem}[Problem sortowania]\label{problem:sorting}
    Niech \( D \) będzie niepustym zbiorem 
    i~\( A \left[ 1 \isep n \right] \) będzie 
    tablicą \( n \)-elementową (\( n \in \NN \)) o~elementach
    ze~zbioru~\( D \).
    Niech porządek \( \le \) będzie porządkiem liniowym 
    na~elementach ciągu \( A \).
    Należy znaleźć taką permutację 
    \( \sigma : \{ 1, 2, \dotsc, n \} \to \{ 1, 2, \dotsc, n \} \), 
    że~\( a_{\sigma(1)} \le a_{\sigma(2)} \le a_{\sigma(3)} 
    \le \dotso \le a_{\sigma(n - 1)} \le a_{\sigma(n)} \).
\end{problem}

\begin{remark}
    W szczególności będziemy mówić krótko, że \( n \)-elementowa tablica \( A \) 
    jest posortowana wtedy, 
    gdy~\( a_1 \le a_2 \le a_3 \le \dotso \le a_{n - 1} \le a_n \).
\end{remark}


Szeroką klasą algorytmów sortowania rozwiązujący problem sortowania
są~algorytmy, które wykorzystują pewien porządek liniowy \( \le \),
by uzyskać żądaną permutację. Jednym z nich, który stosunkowo
jest prosty w implementacji jest sortowanie przez wstawianie.

\section{Sortowanie przez wstawianie}

Sortowanie przez wstawianie intuicyjnie odbywa~się w~taki sposób,
w~jaki ludzie układają karty. W~każdym kroku układania tych kart 
zostaje wstawiona karta do podtablicy już posortowanych kart.
Jak przekonamy się później, ten algorytm łatwo zaimplementować.

\subsection{Algorytm wstawiania}
Omówmy sobie algorytm, który odpowiada za~działanie 
\textsc{InsertionSort}a. Jest to~algorytm wstawiania 
do~posegregowanej podtablicy \( A[1 \isep i] \) \(i\)-tego elementu
tablicy \(A[1 \isep n]\), który będziemy określać
mianem \textsc{Insert}. 
\begin{algorithm}%
    \label{alg:insert}
    \caption{Wstawianie do posortowanej podtablicy}
    \begin{algorithmic}[1]
        \Require The array \( A[1 \isep i - 1] \) is already sorted.
        \Ensure The array \( A[1 \isep i] \) is already sorted.
        \Procedure{Insert}{$A[1 \isep i]$}
            \State \( \text{key} \gets A[i]\)
            \State \( j \gets i \)
            \While{\( j \ge 1 \wedge A[j] > \text{key} \)}\label{alg:insert:while}
                \State \( A[j + 1] \gets A[j] \)
                \State \( j \gets j - 1 \)
            \EndWhile
            \State \( A[j + 1] \gets \text{key} \)
        \EndProcedure%
    \end{algorithmic}
\end{algorithm}
Udowodnijmy ważny niezmiennik.
\begin{invariant}\label{inv:insert}
    Dla~każdej iteracji pętli (w~linii~\ref{alg:insert:while})
    \( 0 \le j_{\text{min}} \le j \le i - 1 \) podtablica \( A[ j + 1 \isep i] \)
    zawiera elementy większe od~klucza,
    przy czym \( j_{\text{min}} = \min \{ k : A[k] > \text{key} \} \).
\end{invariant}
\begin{proof}
    Indukcja względem iteracji.
    Załóżmy, że~\( j_{\text{min}} \le i - 1 \). W~wypadku,
    gdy~\( j_{\text{min}} > i - 1 \) klucz jest wstawiony trywialnie,
    bowiem jest~on elementem maksymalnym w~rozpatrywanej podtablicy.
    Sprawdźmy tezę dla~\( j = i - 1 \). 
    Ponieważ \( j_{\text{min}} \le j \)
    oraz tablica \( A[1 \isep i - 1] \) jest posortowana, 
    to~\( A[j_{\text{min}}] > \text{key} \) i~następnie
    \( A[j_{\text{min}}] < A[i - 1] \), 
    czyli \( A[i - 1] > \text{key} \). 
    Zatem po wykonaniu tego kroku mamy, że~\( A[i] \gets A[i - 1] \),
    czyli otrzymujemy, że~\( A[i] > \text{key} \),
    innymi słowy, tablica \( A[i \isep i] \) spełnia tezę.
    
    Załóżmy teraz, że dla~pewnego \( k \) takiego, 
    że~\( j_{\text{min}} \le k \le i - 1 \),
    podtablica \( A[ k + 1 \isep i ] \)
    jest posortowana i~zawiera elementy większe od~klucza.
    Niech \( k' = k - 1 \) takie, że~\( j_{\text{min}} \le k' \).
    Ponieważ \( j_{\text{min}} \le k' \) 
    oraz tablica \( A[1 \isep i] \) jest posortowana, 
    to~\( A[j_{\text{min}}] > \text{key} \) i~następnie
    \( A[j_{\text{min}}] < A[k'] \), czyli 
    \( A[k'] > \text{key} \). Wobec tego tablica zostanie przesunięta
    o~jeden element, to~znaczy, 
    \( A = (a_1, a_2, \dotsc, a_{j_{\text{min}}},
    a_{j_{\text{min}} + 1}, \dotsc, 
    a_{k' - 1}, a_{k'}, a_{k'}, a_{k' + 1},
    \dotsc, a_{i}, a_{i + 1}, \dotsc, a_{n - 1}, a_n) \).
    Z~założenia indukcyjnego podtablica \( A[k + 1 \isep i] \)
    ma~elementy większe od~klucza.
    Ponieważ \( A[k] > \text{key} \), to~\( A[k \isep i] 
    = A[k' + 1 \isep i] \)
    ma~elementy większe od~klucza, co~należało pokazać.
\end{proof}

\subsection{Algorytm sortowania przez wstawianie}
Mając udowodniony algorytm wstawiania, możemy w~prosty sposób
ułożyć algorytm sortowania przez wstawianie.

\begin{algorithm}%
    \label{alg:insertion-sort}
    \caption{Sortowanie przez wstawianie}
    \begin{algorithmic}[1]
        \Require \( n \in \{ k \in \NN : k \ge 1 \} \)
        \Ensure The array \( A[1 \isep n] \) is already sorted.
        \Procedure{InsertionSort}{$A[1 \isep n]$}
            \For{\( i \gets 2 \isep n \)}
                \Call{Insert}{$A[1 \isep i]$}
            \EndFor
        \EndProcedure%
    \end{algorithmic}
\end{algorithm}

\begin{invariant}
    Dla każdego kroku \( 2 \le i \le n \), tablica \( A[1 \isep i] \)
    jest posortowana.
\end{invariant}
\begin{proof}
    Dzięki niezmiennikowi~\ref{inv:insert}, 
    po~wykonaniu algorytmu~\textsc{Insert} otrzymujemy,
    iż~\( j = j_{\text{min}} - 1 \) oraz podtablica 
    \( A[ j_{\text{min}} + 1 \isep i] \)
    zawiera elementy większe od~klucza. Ponieważ
    tablica \( A[1 \isep i - 1] \) jest posortowana,
    to~\( A[j_{\text{min}} - 1] 
    \le A[j_{\text{min}}] \le \text{key} \) i~stąd
    wstawiamy \( A[j + 1] \gets \text{key} \).
    Mamy wobec tego, że~tablica \( A[1 \isep i] \) 
    jest posortowana, ponieważ \( A[j_{\text{min}} - 1] 
    \le \text{key} \le A[j_{\text{min}} + 1] \).
\end{proof}

Dzięki temu algorytm sortowania przez wstawianie 
sprawia, że tablica \( A[1 \isep n] \) jest posortowana.

\subsection{Analiza złożoności sortowania przez wstawianie}
Wykonajmy analizę złożoności pesymistycznej.
\begin{fact}
    Czas działania algorytmu~\ref{alg:insert}
    jest rzędu \( \BigO(i) \).
\end{fact}
\begin{proof}
    Niech tablica \( A[1 \isep i] \) będzie 
    posortowana odwrotnie, to~znaczy
    \( a_1 > a_2 > \dotso > a_{i - 1} > a_i \).
    Stąd pętla (w~linii~\ref{alg:insert:while})
    wykona~się \( (i - 1) - j_{\text{min}} + 1 \)
    razy, przy czym \( j_{\text{min}} = 1 \),
    czyli \( i - 1 \) razy. Stąd widzimy,
    że \( i - 1 = \BigO(i) \).
\end{proof}

\begin{fact}
    Czas działania algorytmu~\ref{alg:insertion-sort}
    jest rzędu \( \BigO(n^2) \).
\end{fact}
\begin{proof}
    Niech tablica \( A[1 \isep n] \) będzie 
    posortowana odwrotnie, to znaczy
    \( a_1 > a_2 > \dotso > a_{n - 1} > a_n \).
    Stąd procedura~\ref{alg:insert}
    wykona się \( n - 1 \).
    Stąd widzimy,
    że \( \sum_{i = 2}^n \BigO(i)
     = \BigO \left( \frac{n(n + 1)}{2} \right)
     = \BigO(n^2) \).
\end{proof}

Jak można zauważyć, algorytm jest niewydajny dla~dużych
danych dzięki złożoności kwadratowej. Również pokażemy,
że~dla przypadku średniego nadal nie~otrzymujemy
lepszej złożoności. Zanim przejdziemy do~analizy,
przypomnijmy sobie z~algebry następujące definicje.

\begin{definition}
    Permutacją \( \pi \) skończonego zbioru \( A \)
    jest bijekcja \( \pi : A \to A \).
\end{definition}

\begin{definition}
    Niech \( (A, \le) \) będzie porządkiem liniowym.
    Inwersją dla permutacji \( \pi : A \to A \) 
    nazywamy parę \( (i, j) \in A \times A \)
    taką, że \( i < j \) oraz \( \pi(i) > \pi(j) \).
\end{definition}

Czytelnik może~się zastanawiać, czemu potrzebne~są~nam 
inwersje w~analizie złożoności średniej.
Należy zauważyć, że 
\begin{observation}
    Po wykonaniu algorytmu~\ref{alg:insert} 
    liczba inwersji zmniejsza~się o~jeden.
\end{observation}      
Dowód tej~obserwacji pozostawiamy jako ćwiczenie.
Stąd można spojrzeć na~inwersje istniejące w~tablicy,
która zostanie posortowana.

\begin{fact}
    Średni czas działania algorytmu~\ref{alg:insertion-sort}
    wynosi \( \BigTheta (n^2) \).
\end{fact}
\begin{proof}
    Niech \( A = (a_1, a_2, \dotsc, a_n) \) będzie
    tablicą oraz, \( \pi \) będzie permutacją
    taką, że~\( a_{\pi(1)} \le a_{\pi(2)} 
    \le \dotsc \le a_{\pi(n)} \).
    Niech \( I_{i,j} \) będzie zmienną losową, która 
    jest zadana wzorem.
    \begin{equation*}
        I_{i,j} = \begin{cases}
            1 & i < j \wedge \pi(i) > \pi(j) \\
            0 & \text{w p.w.}
        \end{cases}
    \end{equation*}
    Niech \( I = \sum_{i < j} I_{i, j} \).
    Z~liniowości wartości oczekiwanej mamy 
    \( \Ex [I] = \Ex \left[ \sum_{i < j} I_{i, j} \right] 
    = \sum_{i < j} \Ex [I_{i, j}] \).

    Zauważmy, że~jeśli permutacja \( \pi \)
    ma~inwersję \( i < j \), to~można przyporządkować 
    \begin{equation*}
        (\pi(1), \dotsc, \pi(i), \dotsc \pi(j), \dotsc, 
        \pi(n)) \mapsto 
        (\pi(1), \dotsc, \pi(j), \dotsc \pi(i), \dotsc, 
        \pi(n))
    \end{equation*}
    przy czym
    to~przyporządkowanie jest bijekcją.
    Stąd otrzymujemy, że permutacji z~inwersją \( i < j \)
    jest tyle samo, co~bez inwersji, czyli możemy
    natrafić na~nią z~jednakowym prawdopodobieństwem
    \( \PP [I_{i, j} = 1] = \frac{1}{2} \) dla dowolnych
    \( i < j \). Oczywiście \( \Ex [I_{i, j}] 
    = \PP [I_{i, j} = 1] = \frac{1}{2} \).
    Uporządkowanych par \( (i, j) \) możemy 
    ustalić na~\( \binom{n}{2} \) sposobów,
    ponieważ wystarczy wybrać podzbiór dwuelementowy
    i~z~tego, że~mamy liniowy porządek, możemy
    utworzyć ciąg rosnący.
    Wobec tego mamy, 
    że~\( \Ex [I] = \binom{n}{2} \cdot \frac{1}{2} 
    = \frac{n(n - 1)}{4} = \BigTheta (n^2) \).
\end{proof}

To, że~algorytm ma~złożoność kwadratową, nie~oznacza,
że~jest nieprzydatny. Biegły Czytelnik zawuaży, 
że~złożoność pamięciowa tego algorytmu jest 
rzędu \( \BigO (1) \), czyli jest tani w~kwestii
użycia pamięci. Zatem algorytm nadaje~się
do~małych próbek danych oraz w~technikach
hybrydowych.

