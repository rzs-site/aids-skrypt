\chapter{Algorytmy sortowania}

\epigraph{Jeśli coś jest głupie i działa, to nie jest głupie.}{autor nieznany}

Uważa się, że~jednym z~najbardziej fundamentalnych problemów 
leżących u~podstaw informatyki jest problem sortowania. W~tym rozdziale 
zostaną przedstawione jedne z~najbardziej popularnych algorytmów sortowania.
Zanim jednak podejmiemy~się prób opisu tych algorytmów, przyda~się teoretyczne wprowadzenie.

\section{Wprowadzenie}\label{sec:sorting:intro}

Wprowadźmy ważne definicje.

\begin{definition}[Tablica]\label{def:array}
    Niech \( D \) będzie niepustym zbiorem. Tablicą
    \( n \)-elementową (\( n \in \NN \)) o elementach
    ze zbioru \( D \) nazywamy skończony ciąg
    \( A = (a_1, a_2, a_3, \dotsc, a_{n - 1}, a_n) \in D^n \).
\end{definition}
\begin{remark}
    W~przypadku podciągu \( (a_i, a_{i + 1}, a_{i + 2}, \dotsc, a_{j - 1}, a_j) \) 
    stosujemy oznaczenie \( A \left[ a \isep b \right] \),
    przy~czym \( 1 \le i \le n \), \( 1 \le j \le n \) 
    oraz~\( i \le j \).
    W~szczególności \( A \left[ 1 \isep n \right] \) 
    oznacza całą tablicę \( n \)-elementową.
    Stosować będziemy również oznaczenie na~\( i \)-ty 
    element tej~tablicy jako \(A \left[ i \right] \).
\end{remark}
Mając zdefiniowaną tablicę możemy sformułować następujący problem.
\begin{problem}[Problem sortowania]\label{problem:sorting}
    Niech \( D \) będzie niepustym zbiorem 
    i~\( A \left[ 1 \isep n \right] \) będzie 
    tablicą \( n \)-elementową (\( n \in \NN \)) o~elementach
    ze~zbioru~\( D \).
    Niech porządek \( \le \) będzie porządkiem liniowym 
    na~elementach ciągu \( A \).
    Należy znaleźć taką permutację 
    \( \sigma : \{ 1, 2, \dotsc, n \} \to \{ 1, 2, \dotsc, n \} \), 
    że~\( a_{\sigma(1)} \le a_{\sigma(2)} \le a_{\sigma(3)} 
    \le \dotso \le a_{\sigma(n - 1)} \le a_{\sigma(n)} \).
\end{problem}
